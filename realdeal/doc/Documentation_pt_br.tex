\documentclass[10pt]{article}

\usepackage[english]{babel}
\usepackage[utf8]{inputenc}
\usepackage{graphicx}
\usepackage[a4paper, margin=1in]{geometry}
\usepackage{amsmath}
\usepackage{amssymb}

\title{\textbf{Documentação\\Trabalho Prático - Grafos}\\Projeto e Análise de Algoritmos}
\author{Thiago Vieira de Alcantara Silva\\2017719891}
\date{16 de Outubro de 2017}
\begin{document}

\maketitle

\section{Introdução}
% Introdução com uma explicação clara e objetiva de como o problema que foi resolvido, justificando os algoritmos e as estruturas de dados utilizadas. Para auxiliar nessa atividade utilize pseudocódigos, diagramas e demais figuras que achar conveniente. Não é necessário incluir trechos de código da sua implementação e nem mostrar maiores detalhes da sua implementação, exceto quando esses influenciam no seu algoritmo principal

O problema em que fomos apresentados neste trabalho prático, surge a partir de análises feitas sobre as relações em que deputados mantém no congresso.
Para que pautas sejam aprovadas, ou rejeitadas, deputados tendem a influenciar outros deputados a votarem de acordo com seus interesses.
Sempre podemos estimar o nível de concordância de dois deputados. Com isso, podemos avaliar quais deputados seriam os mais influentes no congresso.
Neste trabalho, seguindo uma modelagem usando teoria dos grafos, definimos matematicamente a influência de um deputado e implementamos uma solução para ordenar os deputados em relação ao seus níveis de influência.

\section{Modelagem do Problema e Solução Proposta}
Para resolver o problema definido acima, começamos modelando a rede de influências de deputados como um grafo $G(V, E)$. No grafo $G$, cada vértice $v \in V$ representa um deputado, cada aresta deste grafo, que liga deputado $u$ a deputado $v$, contém um peso que representa o nível de similaridade dos votos entre os deputados. Assim, temos os limites dos pesos das arestas de $G$: $0 \leq w_e \leq 100 \quad \forall e \in E$.\\
Para resolver o problema de ordenar os deputados por ordem de influência, fizemos uma pequena alteração no grafo. Para aresta, agora temos que: $w_e = 100 - w_e$. Após esta alteração 


\section{Análise de Complexidade}
% TODO: Fazer análise em relação a tempo e espaço.



\section{Análise Empírica}
% Análise experimental que avalie o tempo de execução do seu código em função das características da entrada, tais como, o número de vértices e de arestas. Cabe a você gerar as entradas para esses experimentos. A apresentação dos resultados pode ser feita em gráficos e tabelas que achar conveniente juntamente como as interpretações obtidas neles.


\subsection{Testes}
\subsection{Especificações Técnicas}


\section{Instruções para Utilização do Programa}
\subsection{Compilação}
\subsection{Execução}


\section{Conclusão}


\appendix
\section{Organização dos Arquivos}

\end{document}
