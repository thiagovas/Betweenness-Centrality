\documentclass[10pt]{article}

\usepackage[english]{babel}
\usepackage[utf8]{inputenc}
\usepackage{graphicx}
\usepackage[a4paper, margin=1in]{geometry}
\usepackage{amsmath}
\usepackage{amssymb}
\usepackage{hyperref}
\usepackage{bm}
\usepackage{listings}
\usepackage{url}
\usepackage{forest}


\definecolor{folderbg}{RGB}{124,166,198}
\definecolor{folderborder}{RGB}{110,144,169}

\def\Size{4pt}
\tikzset{
  folder/.pic={
    \filldraw[draw=folderborder,top color=folderbg!50,bottom color=folderbg]
      (-1.05*\Size,0.2\Size+5pt) rectangle ++(.75*\Size,-0.2\Size-5pt);  
    \filldraw[draw=folderborder,top color=folderbg!50,bottom color=folderbg]
      (-1.15*\Size,-\Size) rectangle (1.15*\Size,\Size);
  }
}



\title{\textbf{Documentação\\Trabalho Prático - Grafos}\\Projeto e Análise de Algoritmos}
\author{Thiago Vieira de Alcantara Silva\\2017719891}
\date{16 de Outubro de 2017}
\begin{document}



\renewcommand\refname{Referências}



\maketitle

\section{Introdução}

O problema em que fomos apresentados neste trabalho prático, surge a partir de análises feitas sobre as relações em que deputados mantém no congresso.
Para que pautas sejam aprovadas, ou rejeitadas, deputados tendem a influenciar outros deputados a votarem de acordo com seus interesses.
Sempre podemos estimar o nível de concordância de dois deputados. Com isso, podemos avaliar quais deputados seriam os mais influentes no congresso.
Neste trabalho, seguindo uma modelagem usando teoria dos grafos, definimos matematicamente a influência de um deputado e implementamos uma solução para ordenar os deputados em relação ao seus níveis de influência.

\section{Modelagem do Problema e Solução Proposta}
Para resolver o problema definido acima, começamos modelando a rede de influências de deputados como um grafo $G(V, E)$. No grafo $G$, cada vértice $v \in V$ representa um deputado, cada aresta deste grafo, que liga deputado $u$ a deputado $v$, contém um peso que representa o nível de similaridade dos votos entre os deputados. Assim, temos os limites dos pesos das arestas de $G$: $0 \leq w_e \leq 100 \quad \forall e \in E$.\\
Para resolver o problema de ordenar os deputados por ordem de influência, fizemos uma pequena alteração no grafo. Para cada aresta $e \in E$, agora temos que: $w_e = 100 - w_e$. Após esta alteração, para cada deputado $u$, consideremos a dificuldade de $u$ influenciar um deputado $v$ como sendo o peso do caminho mínimo entre $u$ e $v$. Esta consideração é condizente com o problema proposto, dado que após a modificação de $G$, temos que os pesos das arestas representariam o quão "difícil" seria de um deputado qualquer influenciar o seu vizinho em $G$.\\
Após essas definições, podemos definir o nível de influêcia de um deputado. Dado que um deputado qualquer irá influenciar um outro a partir de deputados intermediários que formem um caminho mínimo até o deputado destino. Definimos a influência de um deputado $v$ como:
$$g(v) = \sum_{s \neq t \neq v} \frac{\sigma_{st}(v)}{\sigma_{st}}$$
Sendo $\sigma_{st}$ o número de caminhos mínimos de $s$ a $t$, $\sigma_{st}(v)$ o número de caminhos mínimos de $s$ a $t$ que passam por $v$.\\
Esta definição é conhecida na literatura como \textit{betweenness centrality}. E por esta medida, tentamos capturar quais deputados são os mais usados como intermediários entre acordos. Assim, para resolver o problema da ordenação de deputados por influência, simplesmente calculamos o \textit{betweenness centrality} de todos os vértices de $G$.


\section{Detalhes de Implementação}
Este trabalho foi desenvolvido usando C++11.
E Para calcular o \textit{betweenness} de cada vértice, utilizamos o algoritmo de Brandes\cite{brandes2001faster}. Devido ao risco de erros de precisão, durante este trabalho procuramos sempre usar variáveis \textit{long long int} e \textit{long double} ( que pode chegar a ter 80 bits )\cite{longDoubleWiki}, além de evitar fazer operações com números de magnitudes extremamente divergentes.

\section{Análise de Complexidade}
No algoritmo de Brandes, basicamente rodamos um algoritmo de Dijkstra para cada vértice $s \in V$, calculando a matrix $\sigma$ e o vetor de dependências $\delta$. Todos estes cálculos são feitos à medida que o algoritmo de Dijkstra é executado. E o custo de cada operação desta é $O(1)$ para cada loop.\\
Ao final de cada execução do Dijkstra, calculamos o vetor de dependências $\delta$ em $O(|E|)$.\\
Assim, como utilizamos a $priority\_queue$ da biblioteca padrão de C++, cada execução do Dijkstra é executada em $O(|E| + |E| * log |V|)$, logo $O(|E| * log |V|)$.\\
Como executamos $|V|$ vezes o algoritmo de Dijkstra, temos que o custo em relação ao tempo para calcular o valor de \textit{betweenness} de cada vértice é $O(|E| * |V| * log |V|)$. Todas as outras operações, como tratamento da entrada ou ordenação do vetor resposta são feitas em no máximo $O(|E| * log |V|)$, assim temos a complexidade final de $\bm{O(|E| * |V| * log |V|)}$.\\

Em relação ao espaço gasto temos o seguinte:\\
Para economizar espaço, representamos somente uma linha da matriz $\sigma$ por execução de Dijkstra, gastando assim somente $O(|V|)$ de memória. Para representarmos o grafo da entrada, assim como o grafo intermediário dos caminhos mínimos, utilizamos uma lista de adjacência, gastando $O(|V| + |E|)$. Todos os outros vetores e maps da nossa implementação, como o vetor resposta, o vetor de distâncias do Dijkstra, o vetor de dependências e os maps usados no tratamento da entrada gastam $O(|V|)$ de espaço. Assim temos a complexidade final em relação a espaço de $\bm{O(|V| + |E|)}$.


\section{Análise Empírica}
Decidimos fazer uma análise empírica seguindo o modelo usado por Brandes, pois assim podemos fazer paralelos mais facilmente.
Para cada configuração de testes, executamos 20 replicações e aqui reportamos a média aritimética delas.
Os testes foram executados em um PCPCPCPCPCPCPCPPCPCPCPCPCPPC.




\section{Instruções para Utilização do Programa}
\subsection{Compilação}
Para compilar o programa, foram disponibilizadas duas opções:\\
(1) \textit{Makefile}, que pode ser executado apenas executando:
\begin{lstlisting}[language=bash]
  $ make
\end{lstlisting}

\noindent
(2) \textit{compilar.sh}, um script bash que compila o código. Para executá-lo:
\begin{lstlisting}[language=bash]
  $ bash compilar.sh
\end{lstlisting}

\noindent
Ou ainda:
\begin{lstlisting}[language=bash]
  $ ./compilar.sh
\end{lstlisting}
Se as permissões de execução de \textit{compilar.sh} estiverem ativadas.\cite{scriptPermissionsStack}


\subsection{Execução}
O programa usa a entrada e saída padrões. Assim, há duas opções de execução do programa:\\
(1) Executá-lo direto da pasta \textit{bin} e redirecionar a entrada e saída:
\begin{lstlisting}[language=bash]
  $ ./bin/main < in > out
\end{lstlisting}

\noindent
(2) \textit{executar.sh}, um script bash que recebe o nome do arquivo de entrada e o de saída e executa o programa:
\begin{lstlisting}[language=bash]
  $ bash executar.sh in out
\end{lstlisting}

\noindent
Ou ainda:
\begin{lstlisting}[language=bash]
  $ ./executar.sh in out
\end{lstlisting}
Se as permissões de execução de \textit{executar.sh} estiverem ativadas.\cite{scriptPermissionsStack}



\section{Conclusão}

Neste trabalho modelamos o problema de ordenar um conjunto de deputados de acordo com seus níveis de influência. Para isto, utilizamos a teoria dos grafos para modelar a rede de influências, e definimos o nível de influência de cada deputado a partir da definição de \textit{betweenness centrality}. Para calcular o valor de \textit{betweenness} de cada vértice, utilizamos o algoritmo de Brandes. Executamos testes empíricos e obtemos resultados satisfatórios em relação ao tempo de execução.


\appendix
\section{Organização dos Arquivos}
 \begin{forest}
      for tree={
        font=\ttfamily,
        grow'=0,
        child anchor=west,
        parent anchor=south,
        anchor=west,
        calign=first,
        inner xsep=7pt,
        edge path={
          \noexpand\path [draw, \forestoption{edge}]
          (!u.south west) +(7.5pt,0) |- (.child anchor) pic {folder} \forestoption{edge label};
        },
        % style for your file node 
        file/.style={edge path={\noexpand\path [draw, \forestoption{edge}]
          (!u.south west) +(7.5pt,0) |- (.child anchor) \forestoption{edge label};},
          inner xsep=2pt,font=\small\ttfamily
                     },
        before typesetting nodes={
          if n=1
            {insert before={[,phantom]}}
            {}
        },
        fit=band,
        before computing xy={l=15pt},
      }  
    [\path{\ }
      [compilar.sh, file
      ]
      [executar.sh, file
      ]
      [Makefile, file
      ]
      [doc
        [Description\_pt\_br.pdf,file
        ]
        [Documentation\_pt\_br.bib,file
        ]
        [Documentation\_pt\_br.tex,file
        ]
        [Documentation\_pt\_br.pdf,file
        ]
      ]
      [src
        [graph.cc,file
        ]
        [graph.h,file
        ]
        [inputter.cc,file
        ]
        [inputter.h,file
        ]
        [main.cc,file
        ]
      ]
    ]
 \end{forest}


\bibliographystyle{plain} % Plain referencing style
\bibliography{Documentation_pt_br} % Use the example bibliography file sample.bib

\end{document}
